\Introduction
Матрица - это прямоугольный массив элементов, записанных в виде набора строк и столбцов, количество которых определяют размер матрицы. В данной лабораторной работе мы рассмотрим несколько алгоритмов умножения матриц. Если говорить о том, где используется умножение матриц, то основное применение данная операция находит в алгоритмах компьютерной графики при операциях над объектами в трёхмерном пространстве. Помимо этого матрицы активно используются в математике, физике, химии и прочих науках для различных вычислений, связанных с массивами данных.

Целью данной лабораторной работы является изучение алгоритмов умножения матриц, получения практических навыков при реализации данных алгоритмы, приобритение навыков расчёта трудоёмкости алгоритмов и получение навыков оптимизации реализации алгоритмов. Для того, чтобы достичь поставленной цели нам необходимо выполнить следующие задачи:
\begin{enumerate} 
	\item Провести анализ данных алгоритмы умножения матриц:
		\begin{enumerate} 
			\item Классический алгоритм умножения матриц;
			\item Алгоритм Винограда для умножения матриц;
		\end{enumerate}
	\item оптимизировать алгоритм Винограда для умножения матриц;
	\item описать используемые структуры данных;
	\item привести схемы рассматриваемых алгоритмов;
	\item вычислить трудоёмкость для указанных выше алгоритмов;
	\item программно реализовать данные выше алгоритмы;
	\item провести сравнительный аналихз каждого алгоритма по затрачиваемому в процессе работы времени.
\end{enumerate}