\chapter{Аналитическая часть}

\section{Цель и задачи}
\textbf{Цель} данной работы заключается в разработке многопоточной конвейерной системы обработки данных.

Требуется решить следующие задачи.
\begin{enumerate}
	\item Описать принципы работы и особенности применения системы обработки данных по конвейерному принципу.
	\item Разработать алгоритмы для стадий обработки данных.
\end{enumerate}

\section{Теоретическая основа}
Конвейерная обработка данных — это техника, при которой общий процесс обработки разделяется на несколько последовательных этапов, причем каждый этап способен выполняться параллельно в отдельном потоке. Это разделение позволяет каждому этапу начинать обработку своей части данных как только она становится доступной, не дожидаясь завершения предыдущих этапов. Таким образом, данные обрабатываются на каждой стадии.

\section{Вывод}
Конвейерная обработка данных представляет собой инструмент для оптимизации вычислительных процессов, особенно в задачах, связанных с большим объёмом данных. Использование многопоточности позволяет значительно ускорить обработку за счёт параллельного выполнения различных этапов.