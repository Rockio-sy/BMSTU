\chapter{Конструкторская часть}
В данной главе представлены основные конструкторские решения, использованные в процессе разработки программного обеспечения. Раздел включает требования к программному обеспечению, определение типов данных, а также структуру проекта.

\section{Требования к реализации программного обеспечения}
Для успешной реализации и функционирования разработанного программного обеспечения были установлены следующие требования:
\begin{itemize}[label=---]
	\item использование многопоточности для обеспечения параллельной обработки данных на каждой стадии соответствующим обработчиком;
	\item вывод в реальном времени статуса задачи;
	\item поддержка параллельного режима обработки данных стадиями конвейера;
	\item вывод статистику каждой задачи после её обработки конвейерной системой.
\end{itemize}

\section{Описание типов данных и классов}
Выделены следующие классы:
\begin{itemize}[label=---]
	\item \texttt{Task} --- сущность, описывающая обрабатываемую задачу.
	\item \texttt{StageBase} --- абстрактный класс, описывающий стадию обработки данных.
	\item \texttt{TaskGeneratorStage} --- сущность, содержащая логику создания задач и их добавление во входную очередь конвейерной системы.
	\item \texttt{ReaderStage} --- сущность, содержащая логику стадии чтения данных.
	\item \texttt{ParserStage} --- сущность, содержащая логику стадии обработки полученных данных.
	\item \texttt{WriterStage} --- сущность, содержащая логику стадии записи данных в базу данных.
	\item \texttt{AccumulatorStage} --- сущность, содержащая логику анализа каждой задачи после её обработки на всех стадиях.
\end{itemize}

\section{Структура проекта}

Программа разделена на заголовочные файлы (.h), исходные файлы (.cpp) и файл сборки (Makefile). Ниже представлена структура каталогов и файлов проекта.

Каталог \textbf{include} --- каталог для заголовочных файлов:
\begin{itemize}[label=---]
	\item \texttt{stagebase.h} --- интерфейс абстрактного класса \textit{StageBase};
	\item \texttt{taskgeneratorstage.h} --- интерфейс класса \textit{TaskGeneratorStage};
	\item \texttt{readerstage.h} --- интерфейс класса \textit{ReaderStage};
	\item \texttt{parserstage.h} --- интерфейс класса \textit{ParserStage};
	\item \texttt{writerstage.h} --- интерфейс класса \textit{WriterStage};
	\item \texttt{accumulatorstage.h} --- интерфейс класса \textit{AccumulatorStage}.
\end{itemize}
Каталог \textbf{source} --- каталог с исходным кодом и реализацией методов классов:
\begin{itemize}[label=---]
	\item \texttt{main.cpp} --- главный файл программы, содержащий точку входа программы;
	\item \texttt{taskgeneratorstage.cpp} --- реализация методов класса \textit{TaskGeneratorStage};
	\item \texttt{readerstage.cpp} --- реализация методов класса \textit{ReaderStage};
	\item \texttt{parserstage.cpp} --- реализация методов класса \textit{ParserStage};
	\item \texttt{writerstage.cpp} --- реализация методов класса \textit{WriterStage};
	\item \texttt{accumulatorstage.cpp} --- реализация методов класса \textit{AccumulatorStage}.
\end{itemize}
Файл \textbf{makefile} - файл для сборки проекта с использованием утилиты make.
Файл \textbf{plot.py} - логика получения аналитических графиков времени выполнения процессов обработки страниц.

\section{Вывод}
Были представлены конструкторские решения, использованные при создании программного обеспечения для создания конвейерной системы.
