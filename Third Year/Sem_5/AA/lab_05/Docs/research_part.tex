\chapter{Исследовательская часть}
В данном разделе будут анализированы данные, полученные при тестировании системы обработки данных по конвейерному принципу.

\section{Анализ времени обработки задач конвейерной системы}
Данные получены при обработке 150 задач.
\begin{itemize}[label=---]
	\item Среднее время существования задачи: 63002 микросекунды.
	\item Среднее время ожидания задачи в очереди для чтения данных: 3143 микросекунды.
	\item Среднее время ожидания задачи в очереди для обработки данных: 41 микросекунды.
	\item Среднее время ожидания задачи в очереди для записи данных: 55437 микросекунды.
	\item Среднее время обработки задачи на стадии чтения данных: 226 микросекунды.
	\item Среднее время обработки задачи на стадии обработки данных: 79 микросекунды.
	\item Среднее время обработки задачи на стадии записи данных: 3711 микросекунды.
\end{itemize}

\section{Технические характеристики устройства}

Настройки устройства, на котором проводилась разработка программы приведены ниже.

\begin{itemize}[label=---]
	\item Операционная система: Ubuntu 22.04 LTS.
	\item Процессор: Intel(R) Core(TM) i5-1035G1 CPU @ 1.00 ГГц.
	\item Оперативная память: 16 Гб.
\end{itemize}

\section{Вывод}
Анализ времени обработки задач в конвейерной системе показывает значительные различия во времени ожидания задач на разных стадиях. Заметно, что среднее время обработки задачи в стадии записи больше всех, так как происходит обращение к базе данных, именно поэтому следует, что время ожидания задач для их записи также больше чем в остальных очередях.
