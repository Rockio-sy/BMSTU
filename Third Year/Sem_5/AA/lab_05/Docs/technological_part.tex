\chapter{Технологическая часть}
В данном разделе описывается выбор инструментов для реализации программы, включая язык программирования. Также представлены реализации ключевых алгоритмов исследования, описания методов тестирования программы и анализ полученных результатов.

\section{Выбор языка программирования}

Для реализации алгоритмов вычисления редакционного расстояния был выбран язык программирования C++. Выбор обусловлен следующими факторами:

\begin{itemize}[label=---]
	\item {производительность};
	\item {объектно-ориентированный подход};
	\item {наличие стандартной библиотеки шаблонов}.
\end{itemize}

\section{Реализация классов}
Интерфейсы классов \texttt{StageBase}, \texttt{Task}, \texttt{TaskGeneratorStage}, \texttt{ReaderStage}, \texttt{WriterStage}, и \texttt{TaskAccumulatorStage} представлены в приложенных листингах \texttt{\ref{lst:StageBase}}, \texttt{\ref{lst:Task}},  \texttt{\ref{lst:TaskGeneratorStage}}, \texttt{\ref{lst:ReaderStage}},  \texttt{\ref{lst:ParserStage}}, \texttt{\ref{lst:WriterStage}},  \texttt{\ref{lst:TaskAccumulatorStage}}.

\begin{center}
	\captionsetup{justification=raggedright,singlelinecheck=off}
	\begin{lstlisting}[caption={Интерфейс класса StageBase}, label={lst:StageBase}]
class StageBase
{
	public:
	virtual ~StageBase() {}
	
	virtual void process(TaskQueue& inputQueue, TaskQueue& outputQueue) = 0;
	
	virtual void shutdown() = 0;
};
	\end{lstlisting}
\end{center}

\begin{center}
	\captionsetup{justification=raggedright,singlelinecheck=off}
	\begin{lstlisting}[caption={Интерфейс класса Task}, label={lst:Task}]
class Task
{
	public:
	
	static int globalId;
	
	Task(const std::string &filePath_) : filePath(filePath_), creationTime(std::chrono::system_clock::now())
	{
		id = ++globalId;
	}
	
	int getId() const;
	const std::string &getFilePath() const;
	const std::string &getContent() const;
	const std::string &getUrl() const;
	const std::string &getTitle() const;
	const std::vector<Ingredient> &getIngredients() const;
	const std::vector<std::string> &getInstructions() const;
	const std::string &getImageUrl() const;
	
	void setId(int id_);
	void setFilePath(const std::string &path_);
	void setContent(const std::string &content_);
	void setUrl(const std::string &url_);
	void setTitle(const std::string &title_);
	void setIngredients(const std::vector<Ingredient> &ingredients_);
	void setInstructions(const std::vector<std::string> &instructions_);
	void setImageUrl(const std::string &imageUrl_);
	void addIngredient(const Ingredient &ingredient_);
	
	
	std::chrono::system_clock::time_point getCreationTime() const;
	std::chrono::system_clock::time_point getQueueEntryTime(TaskQueueID queueID) const;
	std::chrono::system_clock::time_point getQueueExitTime(TaskQueueID queueID) const;
	std::chrono::system_clock::time_point getStageEntryTime(PipelineStage stageID) const;
	std::chrono::system_clock::time_point getStageExitTime(PipelineStage stageID) const;
	std::chrono::system_clock::time_point getDestructionTime() const;
	
	void markQueueEntry(TaskQueueID queueID);
	void markQueueExit(TaskQueueID queueID);
	void markStageEntry(PipelineStage stageID);
	void markStageExit(PipelineStage stageID);
	void markDestruction();
	
	long long getQueueDuration(TaskQueueID queueID) const;
	long long getStageDuration(PipelineStage stageID) const;
	long long getTaskLifeTime() const;
	
	friend std::ostream &operator<<(std::ostream &os, const Task &task);
	
	private:
	int id;
	std::string filePath;
	std::string content;
	std::string url;
	std::string title;
	std::vector<Ingredient> ingredients;
	std::vector<std::string> instructions;
	std::string imageUrl;
	
	struct Timestamp
	{
		std::chrono::system_clock::time_point entry;
		std::chrono::system_clock::time_point exit;
	};
	
	std::chrono::system_clock::time_point creationTime;
	std::chrono::system_clock::time_point destructionTime;
	
	std::map<TaskQueueID, Timestamp> queueTimes;
	std::map<PipelineStage, Timestamp> stageTimes;
};
	\end{lstlisting}
\end{center}

\begin{center}
	\captionsetup{justification=raggedright,singlelinecheck=off}
	\begin{lstlisting}[caption={Интерфейс класса TaskGeneratorStage}, label={lst:TaskGeneratorStage}]
class TaskGeneratorStage : public StageBase
{
	public:
	
	TaskGeneratorStage(const std::string& directoryPath_, int numThreads_, Logger &logger_)
	: directoryPath(directoryPath_), numThreads(numThreads_), threadsCompleted(0), logger(logger_){}
	
	void process(TaskQueue& inputQueue, TaskQueue& outputQueue) override;
	
	void shutdown() override;
	
	private:
	
	std::string directoryPath;
	int numThreads;
	std::atomic<int> threadsCompleted;
	std::mutex mtx;
	std::condition_variable cv;
	Logger &logger;
	
	void generateTasks(TaskQueue &outputQueue, const std::vector<std::string> &paths, int start, int end);
	
	std::vector<std::string> collectAllFilePaths(const std::string &directoryPath);
};
	\end{lstlisting}
\end{center}

\begin{center}
	\captionsetup{justification=raggedright,singlelinecheck=off}
	\begin{lstlisting}[caption={Интерфейс класса ReaderStage}, label={lst:ReaderStage}]
class ReaderStage : public StageBase
{
	public:
	
	explicit ReaderStage(int numThreads_, Logger &logger_)
	: numThreads(numThreads_), logger(logger_) {}
	
	void process(TaskQueue& inputQueue, TaskQueue& outputQueue) override;
	
	void shutdown() override;
	
	private:
	
	int numThreads;
	Logger &logger;
	
	void readTasks(TaskQueue &inputQueue, TaskQueue &outputQueue);
};
	\end{lstlisting}
\end{center}

\begin{center}
	\captionsetup{justification=raggedright,singlelinecheck=off}
	\begin{lstlisting}[caption={Интерфейс класса ParserStage}, label={lst:ParserStage}]
class ParserStage : public StageBase
{
	public:
	
	explicit ParserStage(int numThreads_, Logger &logger_)
	: numThreads(numThreads_), logger(logger_){}
	
	void process(TaskQueue& inputQueue, TaskQueue& outputQueue) override;
	
	void shutdown() override;
	
	private:
	
	int numThreads;
	Logger &logger;
	
	void parseTasks(TaskQueue& inputQueue, TaskQueue& outputQueue);
	
	void parseTask(Task &task);
	
	Ingredient parseIngredient(const std::string &line);
};
	\end{lstlisting}
\end{center}

\begin{center}
	\captionsetup{justification=raggedright,singlelinecheck=off}
	\begin{lstlisting}[caption={Интерфейс класса WriterStage}, label={lst:WriterStage}]

class WriterStage : public StageBase
{
	public:
	
	WriterStage(const std::string &dbConnStr_, int numThreads_, Logger &logger_)
	: dbConnStr(dbConnStr_), numThreads(numThreads_), logger(logger_) {}
	
	void process(TaskQueue& inputQueue, TaskQueue& outputQueue) override;
	
	void shutdown() override;
	
	private:
	std::string dbConnStr;
	int numThreads;
	Logger &logger;
	
	void writeTasks(TaskQueue& inputQueue, TaskQueue& outputQueue);
};
	\end{lstlisting}
\end{center}

\begin{center}
	\captionsetup{justification=raggedright,singlelinecheck=off}
	\begin{lstlisting}[caption={Интерфейс класса TaskAccumulatorStage}, label={lst:TaskAccumulatorStage}]

class AccumulatorStage : public StageBase
{
	public:
	
	AccumulatorStage(int numThreads_, Logger &logger_)
	: numThreads(numThreads_), logger(logger_) {}
	
	void process(TaskQueue& inputQueue, TaskQueue& outputQueue) override;
	
	void shutdown() override;
	
	private:
	
	int numThreads;
	Logger &logger;
	
	void accumulateTasks(TaskQueue& inputQueue, TaskQueue& outputQueue);
};
	\end{lstlisting}
\end{center}

\section{Тестирование}
Программное обеспечение, реализующее конвейерной системы обработки данных было запущено с реальными данными. Логирование текущего статуса каждой задачи позволило проверять правильность работы системы.

\section{Вывод}
Разработанное программное обеспечение успешно справляется с обработкой данных по конвейерному принципу.
