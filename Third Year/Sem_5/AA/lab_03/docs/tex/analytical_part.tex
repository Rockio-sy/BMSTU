\chapter{Аналитический раздел}

\section{Цель и задачи}
\textbf{Цель} - исследование методов поиска элемента в массиве и их реализация для анализа эффективности.

Требуется решить следующие задачи.
\begin{enumerate}[label=\arabic*)]
	\item {Разработка и реализация следующих алгоритмов:}
	\begin{itemize}[label=---]
		\item алгоритм поиска полным перебором;
		\item алгоритм поиска бинарным поиском.
	\end{itemize}
	\item {Тестирование и анализ алгоритмов}. Произвести тестирование реализованных алгоритмов.
\end{enumerate}


\section{Алгоритм поиска полным перебором}
Алгоритм поиска полным перебором проверяет каждый элемент массива последовательно до тех пор, пока не будет найден искомый элемент или до окончания массива. Его сложность составляет \( O(N) \), где \( N \) - количество элементов в массиве.

\section{Алгоритм поиска бинарным поиском}
Алгоритм поиска элемента бинарным поиском работает на принципе деления массива пополам. Этот метод требует, чтобы массив был отсортирован. Если элемент в середине проверяемого диапазона совпадает с искомым, поиск завершается. Если элемент в середине меньше искомого, поиск продолжается в правой половине массива, иначе — в левой. Это уменьшает количество сравнений в \( \log_2(N) \) раз, где \( N \) — количество элементов. Таким образом, время выполнения бинарного поиска составляет \( O(\log N) \).


\section{Вывод}
Были описаны теоретические основы исследуемых алгоритмов. Алгоритм полного перебора, показывает линейную зависимость времени выполнения от размера массива, что делает его неэффективным для больших данных. Алгоритм бинарного поиска, требующий предварительной сортировки массива, увеличивает скорость поиска за счет логарифмической сложности.
