\chapter{Конструкторский раздел}
В данной главе представлены основные конструкторские решения, использованные в процессе разработки программного обеспечения. Раздел включает требования к программному обеспечению, определение типов данных, и структуру проекта.

\section{Требования к реализации программного обеспечения}
Для успешной реализации и функционирования разработанного программного обеспечения были установлены следующие требования:
\begin{enumerate}[label=\arabic*)]
	\item ввод исходного массива и задание искомого элемента;
	\item возможность выбора алгоритма для поиска выбранного элемента;
	\item поддержка различных режимов работы программы, управляемых через интерфейс командной строки;
	\item возможность вывода результатов тестирования.
\end{enumerate}

\section{Описание алгоритмов}

\imgScale{0.3}{bruteForce}{Алгоритм полного перебора}
\imgScale{0.3}{binarySearchPart01}{Алгоритм бинарного поиска}
\imgScale{0.3}{BinrysearchPart2.drawio.png}{Алгоритм бинарного поиска (продолжение)}
\clearpage


\section{Описание типов данных и классов}
Выделены следующие классы:
\begin{enumerate}[label=\arabic*)]
	\item \texttt{Finder} --- сущность, содержащая алгоримты поиска расстояний;
	\item \texttt{TaskHandler} ---  сущность, содержащая логику управление задачами и режимом выполнения программы;
	\item \texttt{EfficiencyAnalyzer} --- сущность, содержащая логику анализа эффективности алгоритмов.
\end{enumerate}

Используются следующие типы данных: ADD ENUMERATE
\begin{enumerate}[label=\arabic*)]
	\item \texttt{целочисленный тип} --- для описания элементов массива в котором проиводится поиск элемента;
	\item \texttt{одномерный целочисленный массив} --- для хранения набора элементов.
\end{enumerate}

\section{Структура проекта}

Программа разделена на заголовочные файлы (.h), исходные файлы (.cpp) и файл сборки (Makefile). Ниже представлена структура каталогов и файлов проекта


Каталог \textbf{include} --- каталог для заголовочных файлов:
\begin{enumerate}[label=\arabic*)]
	\item \texttt{auxiliar.h} --- декларации вспомогательных функций;
	\item \texttt{finder.h} --- интерфейс класса \textit{Finder};
	\item \texttt{taskhandler.h} --- интерфейс класса \textit{TaskHandler};
	\item \texttt{efficiencyanalyzer.h} --- интерфейс класса \textit{EfficiencyAnalyzer}.
\end{enumerate}
Каталог \textbf{source} - каталог с исходным кодом и реализацией методов классов:
\begin{enumerate}[label=\arabic*)]
	\item \texttt{main.cpp} --- главный файл программы, сожержащего точку входа программы;
	\item \texttt{auxiliar.cpp} --- реализация вспомогательных функций;
	\item \texttt{finder.cpp} --- реализация методов класса \textit{Finder};
	\item \texttt{taskhandler.cpp} --- реализация методов класса \textit{TaskHandler};
	\item \texttt{efficiencyanalyzer.cpp} --- реализация методов класса \textit{EfficiencyAnalyzer}.
\end{enumerate}
Файл \textbf{makefile} --- Файл для сборки проекта с использованием утилиты make.
Файл \textbf{plot.py} --- Логика получения аналитических графиков времени выполнения алгоритмов.


\section*{Вывод}
Описаны схемы алгоритмов, выделенные сущности, используемые типы данных и структура проекта.
