\chapter{Исследовательский раздел}

В данном разделе представлены результаты исследования по анализу эффективности алгоритмов изучаемых алгоритмов поиска элемента в массиве. Также приводятся пример работы программы и описываются технические характеристики устройства, на котором проводились тесты.

\section{Примеры работы программы}
Ниже приведен пример работы программы:
\imgScale{0.5}{workExample}{Пример работы программы}
\clearpage

\section{Исследование работы алгоритмов}

\imgScale{0.7}{bruteforce_histogram}{Зависимость количества сравнений от индекс искомого элемента при использовании алгоритма полного перебора}
\FloatBarrier
\imgScale{0.7}{binarysearch_histogram}{Зависимость количества сравнений от индекс искомого элемента при использовании алгоритма бинарного поиска}
\FloatBarrier
\imgScale{0.7}{binarysearchcompstat_histogram}{Зависимость количества сравнений от индекс искомого элемента при использовании алгоритма бинарного поиска (количество сравнений упорядочены)}
\FloatBarrier

\section{Технические характеристики устройства}

Настройки устройства, на котором проводились тесты, разработка программы и анализ работы алгоритмов следующие:

\begin{enumerate}[label=\arabic*)]
	\item операционная система --- Ubuntu 22.04 LTS;
	\item процессор --- Intel(R) Core(TM) i5-1035G1 CPU @ 1.00 ГГц;
	\item оперативная память --- 16 Гб.
\end{enumerate}

\section{Вывод}
Алгоритм бинарного поиска значительно эффективнее алгоритма полного перебора в терминах количества сравнений, необходимых для нахождения элемента. Полный перебор, имея линейную сложность, становится неэффективным при увеличении размера данных. В свою очередь, бинарный поиск демонстрирует логарифмическую сложность.

