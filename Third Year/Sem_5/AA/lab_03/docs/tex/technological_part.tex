\chapter{Технологический раздел}
В данном разделе описывается выбор инструментов для реализации программы, включая язык программирования. Также представлены реализации ключевых алгоритмов исследования, описания методов тестирования программы и анализ полученных результатов.

\section{Выбор языка программирования}

Для реализации алгоритмов вычисления редакционного расстояния был выбран язык программирования C++. Выбор обусловлен следующими факторами:

\begin{itemize}[label=---]
	\item {производительность;}
	\item {объектно-ориентированный подход;}
	\item {наличие стандартной библиотеки шаблонов.}
\end{itemize}

\section{Реализация алгоритмов}

\begin{center}
	\captionsetup{justification=raggedright,singlelinecheck=off}
	\begin{lstlisting}[caption=Реализация алгоритма поиска полным перебором]
std::pair<int, int> Finder::bruteForce(const std::vector<int> &src, int x)
{
	int comparisions = 0;
	for (unsigned long int i = 0; i < src.size(); i++)
	{
		++comparisions;
		if (src[i] == x)
		{
			return {i, comparisions};
		}
	}
	return {-1, comparisions};
}
	\end{lstlisting}
\end{center}

\begin{center}
	\captionsetup{justification=raggedright,singlelinecheck=off}
	\begin{lstlisting}[caption=Реализация алгоритма бинарного поиска]
std::pair<int, int> Finder::binarySearch(const std::vector<int> &src, int x)
{
	int left = 0;
	int right = src.size() - 1;
	int comparisions = 0;
	
	while (left <= right)
	{
		int mid = left + (right - left) / 2;
		++comparisions;
		
		if (src[mid] == x)
		{
			return {mid, comparisions};
		}
		else if (src[mid] < x)
		{
			left = mid + 1;
		}
		else
		{
			right = mid - 1;
		}
	}
	return {-1, comparisions};
}
	\end{lstlisting}
\end{center}

\section{Тестирование}

Тестирование реализаций алгоритмов проводилось на основе следующих классов эквивалентности:

\begin{enumerate}[label=\arabic*)]
	\item {искомого элемента нет в массиве;}
	\item {искомый элемент находится на первой позиции массива;}
	\item {искомый элемент находится в середине массива;}
	\item {искомый элемент находится на последней позиции массива;}
	\item {искомый элемент находится в произвольной позиции массива.}
\end{enumerate}

Результаты тестирования представлены в таблице:

\begin{table}[h!]
	\centering
	\caption{Результаты тестирования алгоритмов вычисления расстояний}
	\label{tab:test_results}
	\begin{tabular}{|c|c|c|c|c|}
		\hline
		\textbf{№} & \multicolumn{2}{c|}{\textbf{Входные данные}} & \multicolumn{2}{c|}{\textbf{Результат}} \\ \cline{2-5}
		& \textbf{Массив} & \textbf{Элемент} & \textbf{Полный перебор} & \textbf{Бин. поиск} \\ \hline
		1 & [1, 3, 5, 7] & 9 & -1 & -1 \\ \hline
		2 & [2, 4, 6, 8] & 2 & 0 & 0 \\ \hline
		3 & [11, 22, 33, 44, 55] & 33 & 2 & 2 \\ \hline
		4 & [9, 8, 7, 6, 5] & 5 & 4 & 4 \\ \hline
		5 & [1, 4, 8, 2, 3] & 2 & 3 & 3 \\ \hline

	\end{tabular}
\end{table}



\section*{Вывод}

В результате технологического анализа были выбраны инструменты и методы, наиболее подходящие для реализации задачи вычисления редакционных расстояний и были выделены класси эквивалентности для тестирования алгоритмов. Все тесты были пройдены.

