\chapter{Аналитическая часть}

\section{Цель и задачи}
\textbf{Цель} данной работы заключается в демонстрации преимуществ параллельной обработки данных при обработке веб-страниц с использованием нативных потоков.

Основные задачи:
\begin{enumerate}
	\item Изучить принципы работы и особенности применения нативных потоков в операционных системах.
	\item Разработать алгоритмы для последовательного и параллельного обработки данных.
	\item Сравнить производительность последовательного и параллельного обработки веб-страниц на примере выбранных интернет-ресурсов.
	\item Оценить возможные улучшения производительности и эффективности при использовании многопоточности.
\end{enumerate}

\section{Теоретическая основа}
Обработка веб-страниц это процесс автоматического извлечения данных с них. Использование нативных потоков позволяет обрабатывать несколько страниц одновременно, что существенно повышает скорость работы.

Нативные потоки — это потоки, управляемые операционной системой, которые позволяют выполнять несколько задач одновременно. Это достигается за счет распределения задач по ядрам процессора.

\section{Вывод}
Были подставлены задачи и изучены теоретические основы для выполнения нужных алгоритмов для обработки веб-страниц.
