\chapter{Исследовательская часть}
TODO

\section{Время выполнения алгоритмов}
Для получения наиболее точных результатов время выполнения обработки данных каждый алгоритм был запущен 15 раз, после чего полученные результаты были усреднены.

\begin{table}[h]
	\begin{center}
		\begin{threeparttable}
			\captionsetup{justification=raggedright,singlelinecheck=off}
			\caption{\label{tbl:time_res}Результаты замеров времени}
			\begin{tabular}{|c|c|}
				\hline
				Количество потоков &Время (в миллисекундах)\\
				\hline
				1 & 748670\\ 
				\hline
				2 & 349475\\  
				\hline
				4 & 172167\\  
				\hline
				8 & 104511\\ 
				\hline
				16 & 74627\\ 
				\hline
				32 & 61236\\ 
				\hline
				64 & 55549\\ 
				\hline
			\end{tabular}
		\end{threeparttable}
	\end{center}
\end{table}

\imgScale{0.7}{performanceScraping}{Зависимость времени обработки данных веб-страниц от количества потоков}
\FloatBarrier

\section{Технические характеристики устройства}

Настройки устройства, на котором проводились замеры времени и разработка программы, приведены ниже:

\begin{itemize}
	\item Операционная система: Ubuntu 22.04 LTS
	\item Процессор: Intel(R) Core(TM) i5-1035G1 CPU @ 1.00GHz
	\item Оперативная память: 16 Гб.
\end{itemize}

\section{Вывод}
Исходя из результатов исследования, можно наблюдать значительное уменьшение времени обработки данных с увеличением количества используемых потоков. Это подтверждает эффективность применения параллельных вычислений.
