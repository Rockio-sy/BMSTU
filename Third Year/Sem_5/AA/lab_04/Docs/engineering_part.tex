\chapter{Конструкторская часть}
В данной главе представлены основные конструкторские решения, использованные в процессе разработки программного обеспечения. Раздел включает требования к программному обеспечению, определение типов данных, а также структуру проекта.

\section{Требования к реализации программного обеспечения}
Для успешной реализации и функционирования разработанного программного обеспечения были установлены следующие требования:
\begin{itemize}
	\item Использование многопоточности для обеспечения параллельной обработки данных.
	\item Предоставление времени выполнения каждого режима обработки данных веб-страниц для оценки их производительности.
	\item Поддержка последовательного и параллельного режимов обработки данных веб-страниц.
	\item Возможность вывода результатов тестирования.
\end{itemize}

\section{Описание типов данных и классов}
Выделены следующие классы:
\begin{itemize}
	\item \texttt{WebPageDownloader} - сущность, содержашая логику скачиваний исходный код веб-страницы для его дальнейшей обработки.
	\item \texttt{HTMLParser} - сущность, содержащая логику анализа и извлечения данных веб-страницы.
	\item \texttt{WebScraper} - сущность, содержащая логику обработки веб-страницы.
	\item \texttt{TimeMeasurer} - сущность, содержащая логику для проведения замеров времени обработки страниц.
\end{itemize}

Используются следующие типы данных:
\begin{itemize}
	\item \texttt{TaskType} - для описания типа задачи при обработки ссылок веб-страниц.
	\item \texttt{Task} - для описания узла в очереди задач при многопоточном режиме обработки страниц.
	\item \texttt{WebScraperConfig} - для описания конфигурации при создании объектов класса WebScraper.
\end{itemize}

\section{Структура проекта}

Программа разделена на заголовочные файлы (.h), исходные файлы (.cpp) и файл сборки (Makefile). Ниже представлена структура каталогов и файлов проекта:

\begin{itemize}
	\item \textbf{include} - каталог для заголовочных файлов
	\begin{itemize}
		\item \texttt{webpagedownloader.h} - интерфейс класса \textit{WebPageDownloader}.
		\item \texttt{htmlparser.h} - интерфейс класса \textit{HTMLParser}.
		\item \texttt{webscraper.h} - интерфейс класса \textit{WebScraper}.
		\item \texttt{timemeasurer.h} - интерфейс класса \textit{TimeMeasurer}.
	\end{itemize}
	\item \textbf{source} - каталог с исходным кодом и реализацией методов классов.
	\begin{itemize}
		\item \texttt{main.cpp} - главный файл программы, содержащий точку входа программы.
		\item \texttt{webpagedownloader.cpp} - реализация методов класса \textit{WebPageDownloader}.
		\item \texttt{htmlparser.cpp} - реализация методов класса \textit{HTMLParser}.
		\item \texttt{webscraper.cpp} - реализация методов класса \textit{WebScraper}.
		\item \texttt{timemeasurer.cpp} - реализация методов класса \textit{TimeMeasurer}.
	\end{itemize}
	\item \textbf{makefile} - файл для сборки проекта с использованием утилиты make.
	\item \textbf{plot.py} - логика получения аналитических графиков времени выполнения процессов обработки страниц.
\end{itemize}

\section*{Вывод}
Были представлены конструкторские решения, использованные при создании программного обеспечения для обработки веб-страниц.
