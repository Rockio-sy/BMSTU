\chapter*{ВВЕДЕНИЕ}
\addcontentsline{toc}{chapter}{ВВЕДЕНИЕ}

Целью данной лабораторной работы является применение навыков динамического программирования в алгоритмах поиска расстояний Левенштейна и Дамерау-Левенштейна.

Определение расстояния Леввенштейна (редакционного расстояния) основано на понятии «редакционное предписание».

Редакционное предписание – последовательность действий, необходимых для получения второй строки из первой  кратчайшим способом. 

Расстояние Левенштейна – минимальное количество действий (вставка, удаление, замена символа), необходимых для преобразования одной строки в другую. 

Если текст был набран с клавиатуры, то вместо расстояния Левенштейна чаще используют расстояние Дамерау – Левенштейна, в котором добавляется еще одно возможное действие - перестановка двух соседних символов.~\cite{Passenger}

Расстояния Левенштейна и Дамерау – Левенштейна применяются в следующих сферах:
\begin{itemize}
	\item[---] компьютерная лингвистика (автозамена в посиковых запросах, текстовая редактура);
	\item[---] биоинформатика (сравнение генов, хромосом и белков);
	\item[---] нечеткий поиск записей в базах (борьба с мошенниками и опечатками).
\end{itemize}

В рамках выполнения работы необходимо решить следующие задачи: 
\begin{enumerate}[label={\arabic*)}]
	\item изучить расстояния Левенштейна и Дамерау-Левенштейна;
	\item разработать алгоритмы поиска этих расстояний;
	\item реализовать разработанные алгоритмы;
	\item провести сравнительный анализ процессорного времени выполнения реализаций этих алгоритмов;
	\item провести сравнительный анализ затрачиваемой реализованными алгоритмами пиковой памяти.
\end{enumerate}
