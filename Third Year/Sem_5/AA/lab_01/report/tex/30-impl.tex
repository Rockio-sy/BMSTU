\chapter{Технологическая часть}

В данном разделе производится выбор средств реализации, а также приводятся требования к программному обеспечению (ПО), листинги реализованных алгоритмов и тесты для программы.

\section{Требования к ПО}

На вход программе подаются две строки (регистрозависимые), а на выходе должно быть получено искомое расстояние, посчитанное с помощью каждого реализованного алгоритма: для расстояния Левенштейна - итерационный и рекурсивный (с кешем и без), а для расстояния Дамерау-Левенштейна - рекурсивный без кеша. Также необходимо вывести затраченное каждым алгоритмом процессорное время.

\section{Выбор средств реализации}

В качестве языка программирования для реализации данной лабораторной работы был выбран язык Python  \cite{PythonBook}. Он позволяет быстро реализовывать различные алгоритмы без выделения большого времени на проектирование сруктуры программы и выбор типов данных. 

Кроме того, в Python есть библиотека time, которая предоставляет функцию process\_time для замера процессорного времени \cite{process_time_text}.

В качестве среды разработки выбран PyCharm. Он является кросс-платформенным, а также предоставляет удобный и функциональнаый отладчик и средства для рефакторинга кода, что позволяет быстро находить и исправлять ошибки \cite{pycharm}.

\section{Листинги кода}

В листингах \ref{lev_matrix} - \ref{dlev_recursion} представлены реализации рассматриваемых алгоритмов.

\begin{lstlisting}[caption=Функция поиска расстояния Левенштейна с заполнением матрицы расстояний,
	label={lev_matrix}]
	def loven_dist_matrix_classic(str1, str2):
    # Empty string so +1!
    n = len(str1) + 1
    m = len(str2) + 1
    matrix = [[0 for i in range(m)] for j in range(n)]  # Create a matrix of zeros

    # Fill with trivial rules (edit distance to/from empty string)
    for i in range(1, n):
        matrix[i][0] = i  # Deleting all characters from str1
    for j in range(1, m):
        matrix[0][j] = j  # Inserting all characters of str2

    # Fill the rest of the matrix
    for i in range(1, n):
        for j in range(1, m):
            insertion = matrix[i][j - 1] + 1
            deletion = matrix[i - 1][j] + 1
            replacement = matrix[i - 1][j - 1] + int(str1[i - 1] != str2[j - 1])

            # Choose the minimum operation
            matrix[i][j] = min(insertion, deletion, replacement)

    # Optionally print the matrix for debugging
    if DEBUG:
        print('Matrix:')
        for line in matrix:
            print(line)

    return matrix[n - 1][m - 1]  # Return the final Levenshtein distance

\end{lstlisting}


\begin{lstlisting}[caption=Функция рекурсивного алгоритма поиска расстояния Левенштейна без кеширования,
	label={lev_recursion_classic}]
	def loven_dist_recursion_classic(str1, str2):
    # trivial rules
    if not str1:
        return len(str2)
    elif not str2:
        return len(str1)

    insertion = loven_dist_recursion_classic(str1, str2[:-1]) + 1
    deletion = loven_dist_recursion_classic(str1[:-1], str2) + 1
    replacement = loven_dist_recursion_classic(str1[:-1], str2[:-1]) + int(str1[-1] != str2[-1])

    return min(insertion, deletion, replacement)

\end{lstlisting}

\begin{lstlisting}[caption=Функция рекурсивного алгоритма поиска расстояния Левенштейна c кешированием,
	label={lev_recursion_optimized}]
	def loven_dist_recursion_optimized(str1, str2):
    def _loven_dist_recursion_optimized(str1, str2, matrix):
        len1 = len(str1)
        len2 = len(str2)

        # trivial rules
        if not len1:
            matrix[len1][len2] = len2
        elif not len2:
            matrix[len1][len2] = len1
        else:
            # insertion
            if matrix[len1][len2 - 1] == -1:
                _loven_dist_recursion_optimized(str1, str2[:-1], matrix)
            # deletion
            if matrix[len1 - 1][len2] == -1:
                _loven_dist_recursion_optimized(str1[:-1], str2, matrix)
            # replacement
            if matrix[len1 - 1][len2 - 1] == -1:
                _loven_dist_recursion_optimized(str1[:-1], str2[:-1], matrix)

            matrix[len1][len2] = min(matrix[len1][len2 - 1] + 1,
                                     matrix[len1 - 1][len2] + 1,
                                     matrix[len1 - 1][len2 - 1] + int(str1[-1] != str2[-1]))

        return

    # Empty String so it should be +1 !!
    n = len(str1) + 1
    m = len(str2) + 1
    matrix = [[-1 for i in range(m)] for j in range(n)]
    _loven_dist_recursion_optimized(str1, str2, matrix)

    if DEBUG:
        print('Matrix:')
        for line in matrix:
            print(line)

    return matrix[n - 1][m - 1]

\end{lstlisting}

\begin{lstlisting}[caption=Функция рекурсивного алгоритма поиска расстояния Дамерау-Левенштейна,
	label={dlev_recursion}]
	def damerau_loven_dist_recursion(str1, str2):
    # trivial rules
    if not str1:
        return len(str2)
    elif not str2:
        return len(str1)

    insertion = damerau_loven_dist_recursion(str1, str2[:-1]) + 1
    deletion = damerau_loven_dist_recursion(str1[:-1], str2) + 1
    replacement = damerau_loven_dist_recursion(str1[:-1], str2[:-1]) + int(str1[-1] != str2[-1])

    if (len(str1) > 1) and (len(str2) > 1) and (str1[-1] == str2[-2]) and (str1[-2] == str2[-1]):
        xchange = damerau_loven_dist_recursion(str1[:-2], str2[:-2]) + 1
        return min(insertion, deletion, replacement, xchange)
    else:
        return min(insertion, deletion, replacement)

\end{lstlisting}

\section{Тестирование}

В таблице \ref{test} приведены функциональные тесты для алгоритмов вычисления расстояний Левенштейна и Дамерау — Левенштейна (в таблице приняты обозначения: РЛ - алгоритм поиска расстояния Левенштейна, РДЛ - алгоритм поиска рассотяния Дамерау-Левенштейна). Все тесты пройдены успешно.

\begin{table}[h]
	\begin{center}
		\caption{\label{test} Тесты}
		\begin{tabular}{|c|c|c|c|}
			\hline
			&                    & \multicolumn{2}{c|}{\bfseries Ожидаемый результат}    \\ \cline{3-4}\hline
			Строка 1& Строка 2 & РЛ & РДЛ \\ [0.5ex] 
			\hline
			 &  & 0 & 0\\
			\hline
			abc & abc & 0 & 0\\
			\hline
			ab & a & 1 & 1\\
			\hline
			a & ab & 1 & 1\\
			\hline
			see & sea & 1 & 1\\
			\hline
			1234 & 1324 & 2 & 1\\
			\hline
			hello & ehlla & 3 & 2\\
			\hline
			cat & pop & 3 & 3\\
			\hline
			кот & скат & 2 & 2\\
			\hline
		\end{tabular}
	\end{center}
\end{table}


\section*{Вывод}

Был производен выбор средств реализации, реализованы и протестированы алгоритмы поиска расстояний: Левенштейна - итерационный и рекурсивный (с кешем и без), Дамерау-Левенштейна - рекурсивный без кеша
