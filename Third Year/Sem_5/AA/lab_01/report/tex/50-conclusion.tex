\chapter*{ЗАКЛЮЧЕНИЕ}
\addcontentsline{toc}{chapter}{ЗАКЛЮЧЕНИЕ}

В результате выполнения лабораторной работы при исследовании алгоритмов нахождения расстояний Левенштейна и Дамерау-Левенштейна были применены и отработаны навыки динамического программирования.

В ходе выполнения лабораторной работы были выполнены следующие задачи: 
\begin{enumerate}[label={\arabic*)}]
	\item изучены расстояния Левенштейна и Дамерау-Левенштейна;
	\item разработаны и рнализованы алгоритмы поиска этих расстояний;
	\item проведен сравнительный анализ процессорного времени выполнения реализаций этих алгоритмов, а также затрачиваемой ими пиковой памяти;
	\item был подготовлен отчет по лабораторной работе.
\end{enumerate}



\newpage

\chapter*{\large СПИСОК ИСПОЛЬЗОВАННЫХ ИСТОЧНИКОВ} % Smaller font size
\addcontentsline{toc}{chapter}{СПИСОК ИСПОЛЬЗОВАННЫХ ИСТОЧНИКОВ} % Add to the Table of Contents

\begin{enumerate}
    \item В. М. Черненький, Ю. Е. Гапанюк. МЕТОДИКА ИДЕНТИФИКАЦИИ ПАССАЖИРА ПО УСТАНОВОЧНЫМ ДАННЫМ. // Вестник МГТУ им. Н.Э. Баумана. Сер. “Приборостроение”. 2012. Т. 39. С. 31–35.
    
    \item Левенштейн, В. И. Двоичные коды с исправлением выпадений, вставок и замещений символов. – М.: Доклады АН СССР, 1965. Т. 163. С. 845–848.
    
    \item Лутц, Марк. Изучаем Python, том 1, 5-е изд. Пер. с англ. — СПб.: ООО “Диалектика”, 2019. Т. 832.
    
    \item time — Time access and conversions [Электронный ресурс]. Режим доступа: \url{https://docs.python.org/3/library/time.html} (дата обращения: 05.09.2021).
    
    \item memory-profiler [Электронный ресурс]. Режим доступа: \url{https://pypi.org/project/memory-profiler/} (дата обращения: 05.09.2021).
\end{enumerate}



