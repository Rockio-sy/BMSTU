\chapter{Технологическая часть}

В данной части рассматривается выбор средств реализации, описывается структура классов программы и приводится интерфейс программного обеспечения.

\section{Технологическая часть}

Для написания данного курсового проекта был выбран язык Python~\cite{python}.

Для разработки интерфейса и работы с пикселями изображения был выбран фреймворк Qt~\cite{qt-framefork}.

Используемые инструменты обладают полной функциональности для разработки, профилирования и отладки необходимой программы, а также создания графического пользовательского интерфейса.


\section{Структура программы}
Разработанная программа состоит из следующих классов. 
Базовые математические классы:
\begin{itemize}[label*=---]
    \item Mat4x4 -- класс матриц;
    \item Vector3D -- класс векторов трехмерного пространства.
\end{itemize}

Классы для работы с моделями:
\begin{itemize}[label*=---]
    \item Model -- базовый класс моделей;
    \item Wave -- класс модели поверхности жидкости.
\end{itemize}

Вспомогательные классы:
\begin{itemize}[label*=---]
    \item Camera -- класс камеры с возможностью перемещения по сцене;
    \item ZBuffer -- класс z-буффера;
    \item RebderWidget -- класс интерфейса;
    \item EngineBase -- класс для отрисовки треугольников;
    \item Engine3D -- класс для отрисовки моделей.
\end{itemize}


На рисунке~\ref{img:class_diagram} представлена схема разработанных классов.
\img{140mm}{class_diagram}{Cхема классов программы}



\section{Реализация алгоритмов}
На листингах \ref{lst:poligon.wgsl} -- \ref{lst:draw_guro.wgsl} приведены реализации алгоритмов
\newpage
\includelisting
    {poligon.wgsl}
    {Алгоритм построения полигональной сетки}

\newpage

\includelisting
    {normals.wgsl}
    {Алгоритм пересчета нормалей модели в каждый момент времени}

\newpage

\includelisting
    {z_buffer.wgsl}
    {Реализация Z-буффера}

\includelisting
    {light_fong.wgsl}
    {Модель освещения Фонга}

\newpage

\includelisting
    {draw_fong.wgsl}
    {Алгоритм закраски по Фонгу (интерполяция нормалей)}

\newpage
    \includelisting
    {draw_guro.wgsl}
    {Алгоритм закраски по Гуро (интерполяция интенсивностей)}

\newpage

\section{Интерфейс программного обеспечения}

При запуске программы в левой части определен виджет сцены, 
в правой части интерфейса определены разделы для настройки сцены (рисунок~\ref{img:start_ui}).
\img{120mm}{start_ui}{Графический интерфейс программы}

\newpage

Для создания модели пользователю необходимо выбрать параметры начальной скорости капли, её ускорения и плотности жидкости в разделе <<Параметры модели>> (рисунок~\ref{img:params}).
\img{40mm}{params}{Параметры модели}

Далее пользователю необходимо выбрать метод закраски модели (Фонг, Гуро) и указать необходимость вывода полигонов в разделе <<Метод закраски>> (рисунок~\ref{img:draw}).
\img{40mm}{draw}{Метод закраски}

Управление камерой описано в разделе <<Камера>> (рисунок~\ref{img:camera}).
\img{40mm}{camera}{Камера}

\newpage
Для создания модели на сцене необходимо нажать кнопку <<Старт>>. 
Меню с настройками модели будет заблокировано, камера станет дострупна для изменения своего расположения.
Для изменения параметров модели или метода закраски необходимо нажать кнопку <<Стоп>>.

На рисунке~\ref{img:run_ui} приведен пример работы программы.
\img{120mm}{run_ui}{Тестовая сцена}

\section*{Вывод}

В данном разделе были выбраны средства реализации, описаны структуры классов программы, описаны модули, а также рассмотрен интерфейс программы.
