\chapter{Аналитический раздел}

В этом разделе рассматриваются существующие алгоритмы создания изображений, а также производится выбор наиболее подходящих методов для решения поставленной задачи.

\section{Формализация объектов сцены}

Сцена включает в себя следующие элементы:
\begin{itemize}[label*=---]
    \item поверхность жидкости --- трёхмерная модель, которая представлена полигональной сеткой, состоящей из соединённых плоских треугольников;
    \item раковина --- трёхмерная модель, внутри которой находится поверхность жидкости;
    \item капля --- трёхмерная модель, вызывающая волновые колебания на поверхности жидкости;
    \item источник света --- задаётся вектором, определяющим направление освещения;
    \item камера --- описывается своим местоположением и направлением обзора.
\end{itemize}

\section{Анализ методов описания трехмерных моделей}

В компьютерной графике выделяют три основных подхода к описанию трехмерных объектов: каркасные, поверхностные и твердотельные модели.
Эти методы обеспечивают различные способы представления объектов, что позволяет корректно отображать их форму и размеры на сцене.

\subsection{Каркасная модель}
Каркасная модель в трёхмерной графике представляет собой набор вершин и рёбер, которые определяют форму многогранного объекта.
Этот тип моделирования является самым простым и обладает существенными ограничениями, основными из которых являются недостаток данных о гранях между линиями и невозможность различить внутреннюю и внешнюю области твердого тела.
Несмотря на это, каркасные модели экономичны с точки зрения памяти и подходят для решения задач, не требующих высокой детализации.
Ключевым недостатком каркасных моделей является неоднозначное представление формы объекта~\cite{MTM}.

\subsection{Поверхностная модель}
Поверхностная модель в трёхмерной графике описывается через точки, линии и поверхности.
Она основывается на предположении, что объекты ограничены поверхностями, которые отделяют их от окружающей среды.
Основным недостатком такого подхода является отсутствие информации о материале по обе стороны от поверхности~\cite{MTM}.

\subsection{Твердотельная модель}
Твердотельная модель включает данные о распределении материала по обе стороны объекта.
Для корректного описания формы необходимо задавать направление внутренней нормали поверхности.

\subsection{Выбор метода описания модели}

Для решения поставленной задачи наиболее подходящим методом является использование поверхностной модели.

Поверхностная модель описывается при помощи полигональной сетки.
Такая сетка включает вершины, рёбра и грани, которые формируют форму объекта в трёхмерном пространстве.

Для представления поверхности жидкости используется сеточная модель, состоящая из массива точек (вершин), соединённых в треугольники.
На каждом временном шаге для вычисления координат точек поверхности применяется волновое уравнение.
Это позволяет быстро обновлять координаты сетки и их значения, обеспечивая динамическую визуализацию.

\newpage 

\section{Анализ алгоритмов удаления невидимых поверхностей}

Удаление невидимых линий и поверхностей является одной из самых сложных задач в области компьютерной графики.  
Алгоритмы, используемые для удаления невидимых элементов, определяют, какие линии, поверхности или объёмы остаются видимыми для наблюдателя, находящегося в определённой точке пространства.  

\subsection{Алгоритм Робертса}

Алгоритм Робертса применяется в пространстве объектов и работает исключительно с выпуклыми телами.  
Если объект не является выпуклым, его предварительно необходимо разбить на составные выпуклые части~\cite{ROB}.  

Процесс выполнения алгоритма включает четыре этапа:  
\begin{itemize}[label*=---]  
    \item Подготовка исходных данных, которая включает составление матриц тел для каждого объекта сцены;  
    \item Удаление рёбер, скрытых самим телом;  
    \item Удаление рёбер, скрытых другими объектами сцены;  
    \item Удаление линий пересечения объектов, скрытых самими телами или другими телами, связанными отношениями пересечения.  
\end{itemize}  

Основное преимущество алгоритма заключается в высокой точности вычислений.  
Однако он работает только с выпуклыми телами и характеризуется ограничением по вычислительной сложности, которая увеличивается пропорционально квадрату количества объектов.  

\subsection{Алгоритм Варнока}  

Алгоритм Варнока действует в пространстве изображения и предназначен для определения, какие грани или их части видимы на сцене.  
Его принцип основан на разбиении области изображения на более мелкие окна, для которых анализируются связанные многоугольники.  
Полигоны, чья видимость может быть однозначно определена, отображаются на сцене.  

\includeimage
    {shapes}
    {f}
    {!ht}
    {0.50\textwidth}
    {Pазбиения в алгоритме Варнока}


Алгоритм работает рекурсивно, это является его главным недостатком.


\subsection{Алгоритм, использующий Z-буфер}

Алгоритм, основанный на использовании Z-буфера, представляет собой простой и популярный метод удаления невидимых поверхностей.  

В данном подходе используются два буфера: буфер кадра и Z-буфер.  
Буфер кадра отвечает за хранение атрибутов каждого пикселя в пространстве изображения.  
Z-буфер представляет собой буфер глубины, который сохраняет значения глубины для каждого видимого пикселя~\cite{ROB}.  

На начальном этапе Z-буфер инициализируется минимальными значениями, а буфер кадра заполняется атрибутами фона.  
В ходе работы алгоритма каждый новый пиксель сравнивается со значением глубины в Z-буфере.  
Если новый пиксель располагается ближе к наблюдателю, его данные записываются в буфер кадра и обновляют Z-буфер.  

Преимуществом алгоритма является его простота в реализации.  
Трудоёмкость работы увеличивается линейно относительно количества объектов на сцене.  
К недостаткам относится высокая потребность в объёме памяти.  

\subsection{Алгоритм обратной трассировки лучей}  

Алгоритм получил название "обратная трассировка лучей" из-за более эффективного подхода, заключающегося в отслеживании пути лучей от наблюдателя к объекту.  
Согласно законам оптики, наблюдатель воспринимает объект через лучи света, испускаемые источником, которые попадают на объект и отражаются в направлении глаза наблюдателя.  


\includeimage
    {ray}
    {f}
    {!ht}
    {0.70\textwidth}
    {Работы обратной трассировки лучей}

Преимуществами являются возможность использования алгоритма в параллельных вычислительных системах и высокая реалистичность получаемого изображения, но необходимо большое количество вычислений.

\subsection{Выбор алгоритма удаления невидимых поверхностей}
Поверхность жидкости аппроксимируется треугольниками. 
Для визуализации поверхности жидкости за основу был взят 
алгоритм построчного сканирования, использующий Z-буфер. 

\newpage 

\section{Анализ алгоритмов закраски}
Методы закраски применяются для создания затенения полигонов в трёхмерной модели с учётом освещения сцены.  
На основе взаимного расположения полигона и источника света определяется уровень освещённости поверхности.  

\subsection{Простая закраска}

Простая закраска подразумевает закрашивание всей грани полигональной модели с одинаковой интенсивностью, рассчитанной по закону Ламберта.  
Этот метод закраски делает поверхность однотонной.  

\includeimage
    {simple_p}
    {f}
    {!ht}
    {0.23\textwidth}
    {Пример простой закраски}

\subsection{Закраска по Гуро}
Метод Гуро устраняет дискретность изменения интенсивности и создает
иллюзию гладкой криволинейной поверхности. Он основан на интерполяции интенсивности. 

\includeimage
    {guro_p}
    {f}
    {!ht}
    {0.25\textwidth}
    {Пример закраски по Гуро}

\subsection{Закраска по Фонгу}
Закраска Фонга по своей идее похожа на закраску Гуро, но ее отличие состоит в том, 
что в методе Гуро по всем точкам полигона интерполируется значения интенсивностей, 
а в методе Фонга --- вектора нормалей, и с их помощью для каждой точки находится значение интенсивности.

\includeimage
    {fong_p}
    {f}
    {!ht}
    {0.25\textwidth}
    {Пример закраски по Фонгу}

\subsection{Выбор алгоритма закраски}
Алгоритм закраски Фонга дает наиболее реалистичное изображение, в частности зеркальных бликов. 
В курсовом проекте будет использоваться метод закраски Фонга и метод закраски Гуро.

\newpage

\section{Анализ моделей освещения}

Существует две основные модели освещения, используемые в компьютерной графике: модель Ламберта и модель Фонга.  

\subsection{Модель освещения Ламберта}

Модель Ламберта описывает идеальное диффузное освещение.  
Уровень освещённости в точке поверхности зависит только от плотности света, падающего на неё,  
и линейно изменяется в зависимости от косинуса угла падения луча.  
При этом направление, в котором смотрит наблюдатель, не влияет на расчёт,  
так как диффузно отражённый свет равномерно рассеивается во всех направлениях.  

Модель Ламберта является одной из самых простых и базовых моделей освещения,  
где рассеянная составляющая света вычисляется согласно закону косинусов (закону Ламберта)~\cite{LIGHT}.  


\includeimage
    {light_lambert}
    {f}
    {!ht}
    {0.35\textwidth}
    {Модель освещения Ламберта}

Все векторы берутся единичными. Тогда косинус угла между ними совпадает 
со скалярным произведением. Формула расчета интенсивности имеет следующий вид: 

\begin{equation}
    I = I_{0} \cdot k_{d} \cdot \cos(\vec{L}, \vec{N}) \cdot I_{d} = I_{0} \cdot K_{d} \cdot (\vec{L}, \vec{N}) \cdot I_{d}
\end{equation}

Где $I$ — результирующая интенсивность света в точке; 
$I_{0}$ — интенсивность источника; 
$k_{d}$ — коэффициент диффузного освещения;  
$\vec{L}$ — вектор от точки до источника; 
$\vec{N}$— вектор нормали в точке; 
$I_{d}$ — мощность рассеянного освещения.


\subsection{Модель освещения Фонга}
Модель представляет собой комбинацию диффузной составляющей и зеркальной составляющей. 
Работает таким образом, что кроме равномерного освещения на материале может также появиться блик. 
Отраженная составляющая в точке зависит от того, насколько близки направления от рассматриваемой 
точки на точку взгляда и отраженного луча. 
Местонахождение блика на объекте, освещенном по модели Фонга, определяется из закона равенства углов падения и отражения. 
Если наблюдатель находится вблизи углов отражения, яркость соответствующей точки повышается~\cite{LIGHT}.

\includeimage
    {light_fong}
    {f}
    {!ht}
    {0.35\textwidth}
    {Модель освещения Фонга}

Для модели Фонга освещение в точке вычисляется по следующей формуле:

\begin{equation}
    I = I_{a} + I_{d} + I_{s}
\end{equation}

Где $I$ --- результирующая интенсивность света в точке; 
$I_{a}$ --- фоновая составляющая; 
$I_{d}$ --- рассеянная составляющая;
$I_{s}$ --- зеркальная составляющая;


Падающий и отраженный лучи лежат в одной плоскости с нормалью к отражающей поверхности в точке падения, и эта нормаль делит угол между лучами на две равные части. 
Таким образом отраженная составляющая освещенности в точке зависит от того, насколько близки направления на наблюдателя и отраженного луча. 
Это можно выразить следующей формулой~\cite{LIGHT}.

Формула для расчета интенсивности для модели Фонга имеет вид:
\begin{equation}
    I = K_a \cdot I_a + I_0 \cdot K_d \cdot (\vec{L}, \vec{N}) \cdot I_d + I_0 \cdot K_s \cdot (\vec{R}, \vec{V})^\alpha \cdot I_s
\end{equation}

Где $I$ --- результирующая интенсивность света в точке; 
$K_a$ --- коэффициент фонового освещения; 
$I_a$ --- интенсивность фонового освещения; 
$I_0$ --- интенсивность источника; 
$K_d$ --- коэффициент диффузного освещения; 
$\vec{L}$ --- вектор от точки до источника; 
$\vec{N}$ --- вектор нормали в точке; 
$I_d$ --- интенсивность диффузного освещения; 
$K_s$ --- коэффициент зеркального освещения; 
$\vec{R}$ --- вектор отраженного луча; 
$\vec{V}$ --- вектор от точки до наблюдателя; 
$\alpha$ --- коэффициент блеска; 
$I_s$ --- интенсивность зеркального освещения.


\subsection{Выбор модели освещения}
Для решения поставленной задачи наиболее оптимальным вариантом является использование модели Фонга в сочетании с закраской по Фонгу.  
Модель освещения Ламберта будет применяться вместе с методом закраски по Гуро.  

\section*{Вывод}

В данном разделе был проведён анализ существующих алгоритмов построения изображений,  
а также выбран наиболее подходящий подход для решения поставленной задачи.  
